\documentclass{article}
\usepackage{fancyvrb}
\usepackage{graphicx}
\usepackage{fullpage}
\usepackage{relsize}
\usepackage{url}
\usepackage{hevea}
\usepackage[shortcuts]{extdash}
\usepackage{textcomp}
% \usepackage{verbdef}

\def\topfraction{.9}
\def\dbltopfraction{\topfraction}
\def\floatpagefraction{\topfraction}     % default .5
\def\dblfloatpagefraction{\topfraction}  % default .5
\def\textfraction{.1}

%HEVEA \footerfalse    % Disable hevea advertisement in footer

\newcommand{\code}[1]{\ifmmode{\mbox{\relax\ttfamily{#1}}}\else{\relax\ttfamily #1}\fi}
%% Hevea version omits "\smaller"
%HEVEA \renewcommand{\code}[1]{\ifmmode{\mbox{\ttfamily{#1}}}\else{\ttfamily #1}\fi}

\newcommand{\includeimage}[2]{
\begin{center}
\ifhevea\imgsrc{#1.png}\else
\resizebox{!}{#2}{\includegraphics{figures/#1}}
\vspace{-1.5\baselineskip}
\fi
\end{center}}

% Add line between figure and text
\makeatletter
\def\topfigrule{\kern3\p@ \hrule \kern -3.4\p@} % the \hrule is .4pt high
\def\botfigrule{\kern-3\p@ \hrule \kern 2.6\p@} % the \hrule is .4pt high
\def\dblfigrule{\kern3\p@ \hrule \kern -3.4\p@} % the \hrule is .4pt high
\makeatother


\title{Annotation File Format Specification}
% Hevea ignores \date, so move the date into \author
\author{\url{http://types.cs.washington.edu/annotation-file-utilities/} \\
\today}
\date{}

\begin{document}

\maketitle

%HEVEA \setcounter{tocdepth}{2}
\tableofcontents

\section{Purpose:  External storage of annotations\label{purpose}}

Java annotations are meta-data about Java program elements, as in
``\code{@Deprecated class Date \{ \ldots\ \}}''.
Ordinarily, Java annotations are written in the source code of a
\code{.java} Java source file.  When \code{javac} compiles the source code,
it inserts the annotations in the resulting \code{.class} file (as
``attributes'').

Sometimes, it is convenient to specify the annotations outside the source
code or the \code{.class} file.
\begin{itemize}
%BEGIN LATEX
\itemsep 0pt \parskip 0pt
%END LATEX
\item
  When source code is not available, a textual file provides a format for
  writing and storing annotations that is much easier to read and modify
  than a \code{.class} file.  Even if the eventual purpose is to insert the
  annotations in the \code{.class} file, the annotations must be specified
  in some textual format first.
\item
  Even when source code is available, sometimes it should not be changed,
  yet annotations must be stored somewhere for use by tools.
\item
  A textual file for annotations can eliminate code clutter.  A developer
  performing some specialized task (such as code verification,
  parallelization, etc.)\ can store annotations in an annotation file without
  changing the main version of the source code.  (The developer's private
  version of the code could contain the annotations, but the developer
  could move them to the separate file before committing changes.)
\item
  Tool writers may find it more convenient to use a textual file, rather
  than writing a Java or \code{.class} file parser.
\item
  When debugging annotation-processing tools, a textual file format
  (extracted from the Java or \code{.class} files) is easier to read and
  is easier for use in testing.
\end{itemize}

All of these uses require an external, textual file format for Java annotations.
The external file format should be easy for people to create, read, and
modify.
%
An ``annotation file'' serves this purpose by specifying a set of
Java annotations.
The Annotation File Utilities
(\url{http://types.cs.washington.edu/annotation-file-utilities/}) are a set
of tools that process annotation files.

The file format discussed in this document supports both standard Java SE 5
declaration annotations and also the type annotations that are introduced by Java SE 8.
The file format provides a simple syntax to represent the structure of a Java
program. For annotations in method bodies of \code{.class} files the annotation
file closely follows
section ``Class File Format Extensions'' of the JSR 308 design document~\cite{JSR308-webpage-201110},
which explains how the annotations are stored in the \code{.class}
file.
In that sense, the current design is extremely low-level, and users
probably would not want to write the files by hand (but they might fill in a
template that a tool generated automatically).  As future work, we should
design a more
user-friendly format that permits Java signatures to be directly specified.
For \code{.java} source files, the file format provides a separate, higher-level
syntax for annotations in method bodies.



%% I don't like this, as it may force distributing logically connected
%% elements all over a file system.  Users should be permitted, but not
%% forced, to adopt such a file structure. -MDE
%   Each file corresponds to exactly one
% ``.class'' file, so (for instance) inner classes are written in
% separate annotation files, named in the same ``{\tt
% OuterClass\$InnerClass}'' pattern as the ``.class'' file.


By convention, an annotation file ends with ``\code{.jaif}'' (for ``Java
annotation index file''), but this is not required.


% \verbdef\lineend|"\n"|

%BEGIN LATEX
\DefineShortVerb{\|}
\SaveVerb{newline}|\n|
\UndefineShortVerb{\|}
\newcommand{\lineend}{\bnflit{\UseVerb{newline}}}
%END LATEX
%HEVEA \newcommand{\bs}{\char"5C}
%HEVEA \newcommand{\lineend}{\bnflit{\bs{}n}}

% literal
\newcommand{\bnflit}[1]{\textrm{``}\textbf{#1}\textrm{''}}
% non-terminal
\newcommand{\bnfnt}[1]{\textsf{\emph{#1}}}
% comment
\newcommand{\bnfcmt}{\rm \# }
% alternative
\newcommand{\bnfor}{\ensuremath{|}}


\section{Grammar\label{grammar}}

This section describes the annotation file format in detail by presenting it in
the form of a grammar. Section~\ref{grammar-conventions} details the conventions
of the grammar. Section~\ref{java-file-grammar} shows how to represent the
basic structure of a Java program (classes, methods, etc.) in an annotation
file. Section~\ref{annotations-grammar} shows how to add annotations to an
annotation file.

\subsection{Grammar conventions\label{grammar-conventions}}

Throughout this document, ``name'' is any valid Java simple name or
binary name, ``type'' is any valid type, and ``value'' is any
valid Java constant, and quoted strings are literal values.
%
The Kleene qualifiers ``*'' (zero or more), ``?'' (zero or one), and ``+''
(one or more) denote plurality of a grammar element.
%
A vertical bar (``\bnfor'') separates alternatives.
Parentheses (``()'') denote grouping, and square brackets (``[]'')
denote optional syntax, which is equivalent to ``( \ldots\ )\ ?''\ but more concise.
We use the hash/pound/octothorpe symbol (``\#'') for comments within the grammar.

In the annotation file,
besides its use as token separator,
whitespace (excluding
newlines) is optional with one exception: no space is permitted
between an ``@'' character and a subsequent name. Indentation is
ignored, but is encouraged to maintain readability of the hierarchy of
program elements in the class (see the example in Section~\ref{example}).

Comments can be written throughout the annotation file using the double-slash
syntax employed by Java for single-line comments: anything following
two adjacent slashes (``//'') until the first newline is a comment.
This is omitted from the grammar for simplicity.
Block comments (``/* \ldots\ */'') are not allowed.

The line end symbol \lineend{} is used for all the different line end
conventions, that is, Windows- and Unix-style newlines are supported.


\subsection{Java file grammar\label{java-file-grammar}}

This section shows how to represent the basic structure of a Java program
(classes, methods, etc.) in an annotation file. For Java elements that can
contain annotations, this section will reference grammar productions contained
in Section~\ref{annotations-grammar}, which describes how annotations are used
in an annotation file.

An annotation file has the same basic structure as a Java program. That is,
there are packages, classes, fields and methods.

The annotation file may omit certain program elements --- for instance, it
may mention only some of the packages in a program, or only some of the
classes in a package, or only some of the fields or methods of a class.
Program elements that do not appear in the annotation file are treated as
unannotated.


\subsubsection{Package definitions\label{package-definitions}}

At the root of an annotation file is one or more package definitions.
A package definition describes a package containing a list of annotation
definitions and classes.  A package definition also contains any
annotations on the package (such as those from a
\code{package-info.java} file).

\begin{tabbing}
\qquad \= \kill
\bnfnt{annotation-file} ::= \\
\qquad    \bnfnt{package-definition}+ \\
\\
\bnfnt{package-definition} ::= \\
\qquad    \bnflit{package} ( \bnflit{:} ) \bnfor{} ( \bnfnt{name} \bnflit{:} \bnfnt{decl-annotation}* ) \lineend \\
\qquad    ( \bnfnt{annotation-definition} \bnfor{} \bnfnt{class-definition} ) *
\end{tabbing}

\noindent
Use a package line of \code{package:} for the default package. Note that
annotations on the default package are not allowed.


\subsubsection{Class definitions\label{class-definitions}}

A class definition describes the annotations present on a class declaration,
as well fields and methods of the class. It is organized according to
the hierarchy of fields and methods in the class.
Note that we use \bnfnt{class-definition} also for interfaces, enums, and
annotation types (to specify annotations in an existing annotation type, not to
be confused with \bnfnt{annotation-definition}s described in
Section~\ref{annotation-definitions}, which defines annotations to be used
throughout an annotation file); for syntactic simplicity, we use \bnflit{class}
for
all such definitions.
% TODO: add test cases for this.

Inner classes are treated as ordinary classes whose names happen to contain
\code{\$} signs and must be defined at the top level of a class definition file.
(To change this, the grammar would have to be extended with a closing
delimiter for classes; otherwise, it would be ambiguous whether a
field or method appearing after an inner class definition belonged to the
inner class or the outer class.) The syntax for inner class names is the same as
is used by the \code{javac} compiler. A good way to get an idea of the inner
class names for a class is to compile the class and look at the filenames of the
\code{.class} files that are produced.

\begin{tabbing}
\qquad \= \kill
\bnfnt{class-definition} ::= \\
\qquad    \bnflit{class} \bnfnt{name} \bnflit{:} \bnfnt{decl-annotation}* \lineend  \\
% TODO: is the order really important? eg. can fields and methods not
% be mixed?
\qquad        \bnfnt{typeparam-definition}* \\
\qquad        \bnfnt{typeparam-bound}* \\
\qquad        \bnfnt{extends}* \\
\qquad        \bnfnt{implements}* \\
\qquad        \bnfnt{field-definition}*  \\
\qquad        \bnfnt{staticinit}* \\
\qquad        \bnfnt{instanceinit}* \\
\qquad        \bnfnt{method-definition}*
\end{tabbing}

\noindent
Annotations on the \bnflit{class} line are annotations on the class declaration,
not the class name.


\paragraph{Type parameter definitions}

The \bnfnt{typeparam-definition} production defines annotations on the
declaration of a type parameter, such as on \code{K} and \code{T} in

\begin{verbatim}
public class Class<K> {
    public <T> void m() {
        ...
    }
}
\end{verbatim}

or on the type parameters on the left-hand side of a member reference,
as on \code{String} in \code{List<String>::size}.

\begin{tabbing}
\qquad \= \kill
\bnfnt{typeparam-definition} ::= \\
\qquad    \bnfcmt The integer is the zero-based type parameter index. \\
\qquad    \bnflit{typeparam} \bnfnt{integer} \bnflit{:} \bnfnt{type-annotation}* \lineend \\
\qquad    \bnfnt{compound-type}*
\end{tabbing}


\paragraph{Type Parameter Bounds}

The \bnfnt{typeparam-bound} production defines annotations on a bound of a
type variable declaration, such as on \code{Number} and \code{Date} in

\begin{verbatim}
public class Class<K extends Number> {
    public <T extends Date> void m() {
        ...
    }
}
\end{verbatim}

\begin{tabbing}
\qquad \= \kill
\bnfnt{typeparam-bound} ::= \\
% The bound should really be a sub-element of the typeparam!
\qquad    \bnfcmt The integers are respectively the parameter and bound indexes of \\
\qquad    \bnfcmt the type parameter bound~\cite{JSR308-webpage-201110}. \\
\qquad    \bnflit{bound} \bnfnt{integer} \bnflit{\&} \bnfnt{integer} \bnflit{:} \bnfnt{type-annotation}* \lineend \\
\qquad    \bnfnt{compound-type}*
\end{tabbing}


\paragraph{Implements and extends}

The \bnfnt{extends} and \bnfnt{implements} productions
define annotations on the names of classes a class \code{extends} or
\code{implements}.
(Note: For interface declarations, \bnfnt{implements} rather than
\bnfnt{extends} defines annotations on the names of extended
interfaces.)

\begin{tabbing}
\qquad \= \kill
\bnfnt{extends} ::= \\
\qquad    \bnflit{extends} \bnflit{:} \bnfnt{type-annotation}* \lineend  \\
\qquad        \bnfnt{compound-type}* \\
\\
\bnfnt{implements} ::= \\
\qquad    \bnfcmt The integer is the zero-based index of the implemented interface. \\
\qquad    \bnflit{implements} \bnfnt{integer} \bnflit{:} \bnfnt{type-annotation}* \lineend  \\
\qquad        \bnfnt{compound-type}*
\end{tabbing}


\paragraph{Static and instance initializers}

The \bnfnt{staticinit} and \bnfnt{instanceinit} productions
define annotations on code within static or instance initializer blocks.

\begin{tabbing}
\qquad \= \kill
\bnfnt{staticinit} ::= \\
\qquad    \bnfcmt The integer is the zero-based index of the implemented interface. \\
\qquad    \bnflit{staticinit} \bnflit{*} \bnfnt{integer} \bnflit{:} \lineend  \\
\qquad        \bnfnt{compound-type}*
\\
\bnfnt{instanceinit} ::= \\
\qquad    \bnfcmt The integer is the zero-based index of the implemented interface. \\
\qquad    \bnflit{instanceinit} \bnflit{*} \bnfnt{integer} \bnflit{:} \lineend  \\
\qquad        \bnfnt{compound-type}*
\end{tabbing}


\subsubsection{Field definitions\label{field-definitons}}

A field definition can have annotations on the declaration, the type of the
field, or --- if in source code --- the field's initialization.

\begin{tabbing}
\qquad \= \kill
\bnfnt{field-definition} ::= \\
\qquad    \bnflit{field} \bnfnt{name} \bnflit{:} \bnfnt{decl-annotation}* \lineend \\
\qquad        \bnfnt{type-annotations}* \\
\qquad        \bnfnt{expression-annotations}*
\end{tabbing}

\noindent
Annotations on the \bnflit{field} line are on the field declaration, not the
type of the field.

The \bnfnt{expression-annotations} production specifies annotations on the
initialization expression of a field. If a field is initialized at declaration
then in bytecode the initialization is moved to the constructor when the class
is compiled. Therefore for bytecode, annotations on the initialization
expression go in the constructor (see Section~\ref{method-definitions}), rather
than the field definition. Source code annotations for the field initialization
expression are valid on the field definition.


\subsubsection{Method definitions\label{method-definitions}}

A method definition can have annotations on the method declaration, in the
method header (return type, parameters, etc.), as well as the method body.

\begin{tabbing}
\qquad \= \kill
\bnfnt{method-definition} ::= \\
\qquad    \bnflit{method} \bnfnt{method-key} \bnflit{:} \bnfnt{decl-annotation}* \lineend \\
\qquad        \bnfnt{typeparam-definition}* \\
\qquad        \bnfnt{typeparam-bound}* \\
\qquad        \bnfnt{return-type}? \\
\qquad        \bnfnt{receiver-definition}? \\
\qquad        \bnfnt{parameter-definition}* \\
% TODO: method throws
\qquad        \bnfnt{variable-definition}* \\
\qquad        \bnfnt{expression-annotations}*
\end{tabbing}

\noindent
The annotations on the \bnflit{method} line are on the method declaration, not
on the return value. The \bnfnt{method-key} consists of the name followed by the
signature in JVML format. For example, the following method

\begin{verbatim}
boolean foo(int[] i, String s) {
  ...
}
\end{verbatim}

\noindent
has the \bnfnt{method-key}:

\begin{verbatim}
foo([ILjava/lang/String;)Z
\end{verbatim}

Note that the
signature is the erased signature of the method and does not contain generic
type information, but does contain the return type. Using \code{javap -s} makes
it easy to find the signature. The method keys ``\code{<init>}'' and
``\code{<clinit>}'' are used to name instance (constructor) and class (static)
initialization methods.

% TODO: exception types in catch clause
% TODO: .class literals
% TODO: type arguments in constructor and method calls


\paragraph{Return type}

A return type defines the annotations on the return type of a method
declaration.  It is also used for the result of a constructor.

\begin{tabbing}
\qquad \= \kill
\bnfnt{return-type} ::=  \\
\qquad    \bnflit{return:} \bnfnt{type-annotation}* \lineend \\
\qquad        \bnfnt{compound-type}*
\end{tabbing}


\paragraph{Receiver definition}

A receiver definition defines the annotations on the type of the receiver
parameter in a method declaration.  A method receiver is the implicit formal
parameter, \code{this}, used in non-static methods.  For source code insertion,
the receiver parameter will be inserted if it does not already exist.

Only inner classes have a receiver.  A top-level constructor does not have
a receiver, though it does have a result.  The type of a constructor result
is represented as a return type.

\begin{tabbing}
\qquad \= \kill
\bnfnt{receiver-definition} ::=  \\
\qquad    \bnflit{receiver:} \bnfnt{type-annotation}* \lineend \\
\qquad    \bnfnt{compound-type}*
\end{tabbing}


\paragraph{Parameter definition}

A formal parameter definition defines the annotations on a method formal
parameter declaration and the type of a method formal parameter, but
\emph{not} the receiver formal parameter.

\begin{tabbing}
\qquad \= \kill
\bnfnt{parameter-definition} ::= \\
\qquad    \bnfcmt The integer is the zero-based index of the formal parameter in the method. \\
\qquad    \bnflit{parameter} \bnfnt{integer} \bnflit{:} \bnfnt{decl-annotation}* \lineend \\
\qquad    \bnfnt{type-annotations}*
\end{tabbing}

\noindent
The annotations on the \bnflit{parameter} line are on the formal parameter
declaration, not on the type of the parameter. A parameter index of 0 is the
first formal parameter. The receiver parameter is not index 0. Use the
\bnfnt{receiver-definition} production to annotate the receiver parameter.


\subsection{Bytecode Locations\label{bytecode-locations}}

Certain elements in the body of a method or the initialization expression of a
field can be annotated. The \bnfnt{expression-annotations} rule describes the
annotations that can be added to a method body or a field initialization
expression:

\begin{tabbing}
\qquad \= \kill
\bnfnt{expression-annotations} ::= \\
\qquad    \bnfnt{typecast}* \\
\qquad    \bnfnt{instanceof}* \\
\qquad    \bnfnt{new}* \\
\qquad    \bnfnt{call}* \\
\qquad    \bnfnt{reference}* \\
\qquad    \bnfnt{lambda}* \\
\qquad    \bnfnt{source-insert-typecast}* \\
\qquad    \bnfnt{source-insert-annotation}*
\end{tabbing}

\noindent
Additionally, a variable declaration in a method body can be annotated with the
\bnfnt{variable-definition} rule, which appears below.

Because of the differences between Java source code and \code{.class} files,
the syntax for specifying code locations is different for \code{.class} files
and source code. For \code{.class} files we use a syntax called ``bytecode
offsets''. For source code we use a different syntax called ``source code
indexes''. These are both described below.

If you wish to be able to insert a given code annotation in both a \code{.class} file and a source
code file, the annotation file must redundantly specify the annotation's bytecode offset and source
code index. This can be done in a single \code{.jaif} file or two separate
\code{.jaif} files. It is not necessary to include
redundant information to insert annotations on signatures in both \code{.class}
files and source code.

Additionally, a new typecast with annotations (rather than an annotation added to an
existing typecast) can be inserted into source code. This uses a third
syntax that is described below under ``AST paths''.
A second way to insert a typecast is by specifying just an annotation, not
a full typecast (\code{insert-annotation} instead of
\code{insert-typecast}).  In this case, the source annotation insertion
tool generates a full typecast if Java syntax requires one.


\subsubsection{Bytecode offsets\label{bytecode-offsets}}

For locations in bytecode, the
annotation file uses offsets into the bytecode array of the class file to
indicate the specific expression to which the annotation refers.  Because
different compilation strategies yield different \code{.class} files, a
tool that maps such annotations from an annotation file into source code must
have access to the specific \code{.class} file that was used to generate
the annotation file.   The
\code{javap -v} command is an effective technique to discover bytecode
offsets.  Non-expression annotations such as those on methods,
fields, classes, etc., do not use a bytecode offset.


\subsubsection{Source code indexes\label{source-code-indexes}}

For locations in source code, the annotation file indicates the kind of
expression, plus a zero-based index to indicate which occurrence of that kind of
expression. For example,

\begin{verbatim}
public void method() {
    Object o1 = new @A String();
    String s = (@B String) o1;
    Object o2 = new @C Integer(0);
    Integer i = (@D Integer) o2;
}
\end{verbatim}

\noindent
\code{@A} is on new, index 0. \code{@B} is on typecast, index 0. \code{@C} is on
new, index 1. \code{@D} is on typecast, index 1.

Source code indexes only include occurrences in the class that exactly matches
the name of the enclosing \bnfnt{class-definition} rule. Specifically,
occurrences in nested classes are not included. Use a new
\bnfnt{class-definition} rule with the name of the nested class for source code
insertions in a nested class.


\subsubsection{Code locations grammar\label{code-grammar}}

For each kind of expression, the grammar contains a separate location rule.
This location rule contains the bytecode offset syntax followed by the
source code index syntax.

The grammar uses \bnflit{\#} for bytecode offsets and \bnflit{*} for source code indexes.

\begin{tabbing}
\qquad \= \kill
\bnfnt{variable-location} ::= \\
\qquad    \bnfcmt Bytecode offset: the integers are respectively the index, start, and length \\
\qquad    \bnfcmt fields of the annotations on this variable~\cite{JSR308-webpage-201110}. \\
\qquad    (\bnfnt{integer} \bnflit{\#} \bnfnt{integer} \bnflit{+} \bnfnt{integer}) \\
\qquad    \bnfcmt Source code index: the \bnfnt{name} is the identifier of the local variable. \\
\qquad    \bnfcmt The \bnfnt{integer} is the optional zero-based index of the intended local \\
\qquad    \bnfcmt variable within all local variables with the given \bnfnt{name}. \\
\qquad    \bnfcmt The default value for the index is zero. \\
\qquad    \bnfor{} (\bnfnt{name} [\bnflit{*} \bnfnt{integer}]) \\
\\
\bnfnt{variable-definition} ::= \\
\qquad    \bnfcmt The annotations on the \bnflit{local} line are on the variable declaration, \\
\qquad    \bnfcmt not the type of the variable. \\
\qquad    \bnflit{local} \bnfnt{variable-location} \bnflit{:} \bnfnt{decl-annotation}* \lineend \\
\qquad    \bnfnt{type-annotations}* \\
\\
\bnfnt{typecast-location} ::= \\
\qquad    \bnfcmt Bytecode offset: the first integer is the offset field and the optional \\
\qquad    \bnfcmt second integer is the type index of an intersection type~\cite{JSR308-webpage-201110}. \\
\qquad    \bnfcmt The type index defaults to zero if not specified. \\
\qquad    (\bnflit{\#} \bnfnt{integer} [ \bnflit{,} \bnfnt{integer} ]) \\
\qquad    \bnfcmt Source code index: the first integer is the zero-based index of the typecast \\
\qquad    \bnfcmt within the method and the optional second integer is the type index of an \\
\qquad    \bnfcmt intersection type~\cite{JSR308-webpage-201110}. The type index defaults to zero if not specified. \\
\qquad    \bnfor{} (\bnflit{*} \bnfnt{integer} [ \bnflit{,} \bnfnt{integer} ]) \\
\\
\bnfnt{typecast} ::= \\
\qquad    \bnflit{typecast} \bnfnt{typecast-location} \bnflit{:} \bnfnt{type-annotation}* \lineend \\
\qquad    \bnfnt{compound-type}* \\
\\
\bnfnt{instanceof-location} ::= \\
\qquad    \bnfcmt Bytecode offset: the integer is the offset field of the annotation~\cite{JSR308-webpage-201110}. \\
\qquad    (\bnflit{\#} \bnfnt{integer}) \\
\qquad    \bnfcmt Source code index: the integer is the zero-based index of the \code{instanceof} \\
\qquad    \bnfcmt within the method. \\
\qquad    \bnfor{} (\bnflit{*} \bnfnt{integer}) \\
\\
\bnfnt{instanceof} ::= \\
\qquad    \bnflit{instanceof} \bnfnt{instanceof-location} \bnflit{:} \bnfnt{type-annotation}* \lineend \\
\qquad    \bnfnt{compound-type}* \\
\\
\bnfnt{new-location} ::= \\
\qquad    \bnfcmt Bytecode offset: the integer is the offset field of the annotation~\cite{JSR308-webpage-201110}. \\
\qquad    (\bnflit{\#} \bnfnt{integer}) \\
\qquad    \bnfcmt Source code index: the integer is the zero-based index of the object or array \\
\qquad    \bnfcmt creation within the method. \\
\qquad    \bnfor{} (\bnflit{*} \bnfnt{integer}) \\
\\
\bnfnt{new} ::= \\
\qquad    \bnflit{new} \bnfnt{new-location} \bnflit{:} \bnfnt{type-annotation}* \lineend  \\
\qquad    \bnfnt{compound-type}*
\\
\bnfnt{call-location} ::= \\
\qquad    \bnfcmt Bytecode offset: the integer is the offset field of the annotation~\cite{JSR308-webpage-201110}. \\
\qquad    (\bnflit{\#} \bnfnt{integer}) \\
\qquad    \bnfcmt Source code index: the integer is the zero-based index of the method call \\
\qquad    \bnfcmt within the field or method definition. \\
\qquad    \bnfor{} (\bnflit{*} \bnfnt{integer}) \\
\\
\bnfnt{call} ::= \\
\qquad    \bnflit{call} \bnfnt{call-location} \bnflit{:} \lineend \\
\qquad    \bnfnt{typearg-definition}* \\
\\
\bnfnt{reference-location} ::= \\
\qquad    \bnfcmt Bytecode offset: the integer is the offset field of the annotation~\cite{JSR308-webpage-201110}. \\
\qquad    (\bnflit{\#} \bnfnt{integer}) \\
\qquad    \bnfcmt Source code index: the integer is the zero-based index of the member \\
\qquad    \bnfcmt reference~\cite{JSR308-webpage-201110}. \\
\qquad    \bnfor{} (\bnflit{*} \bnfnt{integer}) \\
\\
\bnfnt{reference} ::= \\
\qquad    \bnflit{reference} \bnfnt{reference-location} \bnflit{:} \bnfnt{type-annotation}* \lineend \\
\qquad    \bnfnt{compound-type}* \\
\qquad    \bnfnt{typearg-definition}* \\
\\
\bnfnt{lambda-location} ::= \\
\qquad    \bnfcmt Bytecode offset: the integer is the offset field of the annotation~\cite{JSR308-webpage-201110}. \\
\qquad    (\bnflit{\#} \bnfnt{integer}) \\
\qquad    \bnfcmt Source code index: the integer is the zero-based index of the lambda \\
\qquad    \bnfcmt expression~\cite{JSR308-webpage-201110}. \\
\qquad    \bnfor{} (\bnflit{*} \bnfnt{integer}) \\
\\
\bnfnt{lambda} ::= \\
\qquad    \bnflit{lambda} \bnfnt{lambda-location} \bnflit{:} \lineend \\
%\qquad        \bnfnt{return-type}? \\
\qquad        \bnfnt{parameter-definition}* \\
\qquad        \bnfnt{variable-definition}* \\
\qquad        \bnfnt{expression-annotations}*
\\
\qquad \= \kill
\bnfnt{typearg-definition} ::= \\
\qquad    \bnfcmt The integer is the zero-based type argument index. \\
\qquad    \bnflit{typearg} \bnfnt{integer} \bnflit{:} \bnfnt{type-annotation}* \lineend \\
\qquad    \bnfnt{compound-type}*
\end{tabbing}


\subsubsection{AST paths\label{ast-paths}}

A path through the AST (abstract
syntax tree) specifies an arbitrary expression in source code to modify.
AST paths can be used in the \code{.jaif} file to specify a location to
insert either a bare annotation (\bnflit{insert-annotation}) or a cast
(\bnflit{insert-typecast}).

For a cast insertion, the \code{.jaif} file specifies the type to cast to.
The annotations on the \bnflit{insert-typecast} line will be inserted on
the outermost type of the type to cast to. If the type to cast to is a compound
type then annotations on parts of the compound type are specified with the
\bnfnt{compound-type} rule. If there are no annotations on
the \bnflit{insert-typecast} line then a cast with no annotations will be
inserted or, if compound type annotations are specified, a cast with annotations
only on the compound types will be inserted.

Note that the type specified on the \bnflit{insert-typecast} line cannot contain
any qualified type names. For example, use \code{Entry<String, Object>} instead
of \code{Map.Entry<java.lang.String, java.lang.Object>}.

\begin{tabbing}
\bnfnt{source-insert-typecast} ::= \\
\qquad    \bnfcmt \bnfnt{ast-path} is described below. \\
\qquad    \bnfcmt \bnfnt{type} is the un-annotated type to cast to. \\
\qquad    \bnflit{insert-typecast} \bnfnt{ast-path}\bnflit{:} \bnfnt{type-annotation}* \bnfnt{type} \lineend \\
\qquad        \bnfnt{compound-type}*
\end{tabbing}

An AST path represents a traversal through the AST. AST paths can only be
used in \bnfnt{field-definition}s and \bnfnt{method-definition}s.
An AST path starts with the first element under the definition. For
methods this is \code{Block} and for fields this is \code{Variable}.

An AST path is composed of one or more AST entries, separated by commas. Each
AST entry is composed of a tree kind, a child selector, and an optional
argument. An example AST entry is:

\begin{verbatim}
Block.statement 1
\end{verbatim}

The tree kind is \code{Block}, the child selector is \code{statement} and the
argument is \code{1}.

The available tree kinds correspond to the Java AST tree nodes (from the package
\code{com.sun.source.tree}), but with ``Tree'' removed from the name. For
example, the class \code{com.sun.source.tree.BlockTree} is represented as
\code{Block}. The child selectors correspond to the method names of the given
Java AST tree node, with ``get'' removed from the beginning of the method name
and the first letter lowercased. In cases where the child selector method
returns a list, the method name is made singular and the AST entry also contains
an argument to select the index of the list to take. For example, the method
\code{com\-.sun\-.source\-.tree\-.Block\-Tree\-.get\-Statements()} is represented as
\code{Block.statement} and requires an argument to select the statement to take.

The following is an example of an entire AST path:

\begin{verbatim}
Block.statement 1, Switch.case 1, Case.statement 0, ExpressionStatement.expression,
    MethodInvocation.argument 0
\end{verbatim}

Since the above example starts with a \code{Block} it belongs in a
\bnfnt{method-definition}. This AST path would select an expression that is in
statement 1 of the method, case 1 of the switch statement, statement 0 of the
case, and argument 0 of a method call (\code{ExpressionStatement} is just a
wrapper around an expression that can also be a statement).

The following is an example of an annotation file with AST paths used to specify
where to insert casts.

\begin{verbatim}
package p:
annotation @A:

class ASTPathExample:

field a:
    insert-typecast Variable.initializer, Binary.rightOperand: @A Integer

method m()V:
    insert-typecast Block.statement 0, Variable.initializer: @A Integer
    insert-typecast Block.statement 1, Switch.case 1, Case.statement 0,
        ExpressionStatement.expression, MethodInvocation.argument 0: @A Integer
\end{verbatim}

And the matching source code:

\begin{verbatim}
package p;

public class ASTPathExample {

    private int a = 12 + 13;

    public void m() {
        int x = 1;
        switch (x + 2) {
            case 1:
                System.out.println(1);
                break;
            case 2:
                System.out.println(2 + x);
                break;
            default:
                System.out.println(-1);
        }
    }
}
\end{verbatim}

The following is the output, with the casts inserted.

\begin{verbatim}
package p;
import p.A;

public class ASTPathExample {

    private int a = 12 + ((@A Integer) (13));

    public void m() {
        int x = ((@A Integer) (1));
        switch (x + 2) {
            case 1:
                System.out.println(1);
                break;
            case 2:
                System.out.println(((@A Integer) (2 + x)));
                break;
            default:
                System.out.println(-1);
        }
    }
}
\end{verbatim}

Using \code{insert-annotation} instead of \code{insert-typecast} yields
almost the same result --- it also inserts a cast.  The sole difference
is the inability to specify the type in the cast expression.  If you use
\code{insert-annotation}, then the annotation inserter infers the type,
which is \code{int} in this case.

Note that a cast can be inserted on any expression, not
just the deepest expression in the AST. For example, a cast could be inserted on
the expression \code{i + j}, the identifier \code{i}, and/or the identifier \code{j}.

To help create correct AST paths it may be useful to view the AST of a class.
The Checker Framework has a processor to do this. The following command will
output indented AST nodes for the entire input program.

\begin{verbatim}
javac -processor org.checkerframework.common.util.debug.TreeDebug ASTPathExample.java
\end{verbatim}

The following is the grammar for AST paths.

\begin{tabbing}
\qquad \= \kill
\bnfnt{ast-path} ::= \\
\qquad    \bnfnt{ast-entry} [ \bnflit{,} \bnfnt{ast-entry} ]+ \\
\\
\bnfnt{ast-entry} ::= \\
\qquad    \bnfnt{annotated-type} \\
\qquad    \bnfor{} \bnfnt{annotation} \\
\qquad    \bnfor{} \bnfnt{array-access} \\
\qquad    \bnfor{} \bnfnt{array-type} \\
\qquad    \bnfor{} \bnfnt{assert} \\
\qquad    \bnfor{} \bnfnt{assignment} \\
\qquad    \bnfor{} \bnfnt{binary} \\
\qquad    \bnfor{} \bnfnt{block} \\
\qquad    \bnfor{} \bnfnt{case} \\
\qquad    \bnfor{} \bnfnt{catch} \\
\qquad    \bnfor{} \bnfnt{compound-assignment} \\
\qquad    \bnfor{} \bnfnt{conditional-expression} \\
\qquad    \bnfor{} \bnfnt{do-while-loop} \\
\qquad    \bnfor{} \bnfnt{enhanced-for-loop} \\
\qquad    \bnfor{} \bnfnt{expression-statement} \\
\qquad    \bnfor{} \bnfnt{for-loop} \\
\qquad    \bnfor{} \bnfnt{if} \\
\qquad    \bnfor{} \bnfnt{instance-of} \\
\qquad    \bnfor{} \bnfnt{intersection-type} \\
\qquad    \bnfor{} \bnfnt{labeled-statement} \\
\qquad    \bnfor{} \bnfnt{lambda-expression} \\
\qquad    \bnfor{} \bnfnt{member-reference} \\
\qquad    \bnfor{} \bnfnt{member-select} \\
\qquad    \bnfor{} \bnfnt{method-invocation} \\
\qquad    \bnfor{} \bnfnt{new-array} \\
\qquad    \bnfor{} \bnfnt{new-class} \\
\qquad    \bnfor{} \bnfnt{parameterized-type} \\
\qquad    \bnfor{} \bnfnt{parenthesized} \\
\qquad    \bnfor{} \bnfnt{return} \\
\qquad    \bnfor{} \bnfnt{switch} \\
\qquad    \bnfor{} \bnfnt{synchronized} \\
\qquad    \bnfor{} \bnfnt{throw} \\
\qquad    \bnfor{} \bnfnt{try} \\
\qquad    \bnfor{} \bnfnt{type-cast} \\
\qquad    \bnfor{} \bnfnt{type-parameter} \\
\qquad    \bnfor{} \bnfnt{unary} \\
\qquad    \bnfor{} \bnfnt{union-type} \\
\qquad    \bnfor{} \bnfnt{variable-type} \\
\qquad    \bnfor{} \bnfnt{while-loop} \\
\qquad    \bnfor{} \bnfnt{wildcard-tree} \\
\\
\bnfnt{annotated-type} :: = \\
\qquad    \bnflit{AnnotatedType} \bnflit{.} ( ( \bnflit{annotation} \bnfnt{integer} ) \bnfor{} \bnflit{underlyingType} ) \\
\\
\bnfnt{annotation} ::= \\
\qquad    \bnflit{Annotation} \bnflit{.} ( \bnflit{type} \bnfor{} \bnflit{argument} \bnfnt{integer} ) \\
\\
\bnfnt{array-access} ::= \\
\qquad    \bnflit{ArrayAccess} \bnflit{.} ( \bnflit{expression} \bnfor{} \bnflit{index} ) \\
\\
\bnfnt{array-type} ::= \\
\qquad    \bnflit{ArrayType} \bnflit{.} \bnflit{type} \\
\\
\bnfnt{assert} ::= \\
\qquad    \bnflit{Assert} \bnflit{.} ( \bnflit{condition} \bnfor{} \bnflit{detail} ) \\
\\
\bnfnt{assignment} ::= \\
\qquad    \bnflit{Assignment} \bnflit{.} ( \bnflit{variable} \bnfor{} \bnflit{expression} ) \\
\\
\bnfnt{binary} ::= \\
\qquad    \bnflit{Binary} \bnflit{.} ( \bnflit{leftOperand} \bnfor{} \bnflit{rightOperand} ) \\
\\
\bnfnt{block} ::= \\
\qquad    \bnflit{Block} \bnflit{.} \bnflit{statement} \bnfnt{integer} \\
\\
\bnfnt{case} ::= \\
\qquad    \bnflit{Case} \bnflit{.} ( \bnflit{expression} \bnfor{} ( \bnflit{statement} \bnfnt{integer} ) ) \\
\\
\bnfnt{catch} ::= \\
\qquad    \bnflit{Catch} \bnflit{.} ( \bnflit{parameter} \bnfor{} \bnflit{block} ) \\
\\
\bnfnt{compound-assignment} ::= \\
\qquad    \bnflit{CompoundAssignment} \bnflit{.} ( \bnflit{variable} \bnfor{} \bnflit{expression} ) \\
\\
\bnfnt{conditional-expression} ::= \\
\qquad    \bnflit{ConditionalExpression} \bnflit{.} ( \bnflit{condition} \bnfor{} \bnflit{trueExpression} \bnfor{} \bnflit{falseExpression} ) \\
\\
\bnfnt{do-while-loop} ::= \\
\qquad    \bnflit{DoWhileLoop} \bnflit{.} ( \bnflit{condition} \bnfor{} \bnflit{statement} ) \\
\\
\bnfnt{enhanced-for-loop} ::= \\
\qquad    \bnflit{EnhancedForLoop} \bnflit{.} ( \bnflit{variable} \bnfor{} \bnflit{expression} \bnfor{} \bnflit{statement} ) \\
\\
\bnfnt{expression-statement} ::= \\
\qquad    \bnflit{ExpressionStatement} \bnflit{.} \bnflit{expression} \\
\\
\bnfnt{for-loop} ::= \\
\qquad    \bnflit{ForLoop} \bnflit{.} ( ( \bnflit{initializer} \bnfnt{integer} ) \bnfor{} \bnflit{condition} \bnfor{} ( \bnflit{update} \bnfnt{integer} )  \bnfor{} \bnflit{statement} ) \\
\\
\bnfnt{if} ::= \\
\qquad    \bnflit{If} \bnflit{.} ( \bnflit{condition} \bnfor{} \bnflit{thenStatement} \bnfor{} \bnflit{elseStatement} ) \\
\\
\bnfnt{instance-of} ::= \\
\qquad    \bnflit{InstanceOf} \bnflit{.} ( \bnflit{expression} \bnfor{} \bnflit{type} ) \\
\\
\bnfnt{intersection-type} ::= \\
\qquad    \bnflit{IntersectionType} \bnflit{.} \bnflit{bound} \bnfnt{integer} \\
\\
\bnfnt{labeled-statement} ::= \\
\qquad    \bnflit{LabeledStatement} \bnflit{.} \bnflit{statement} \\
\\
\bnfnt{lambda-expression} ::= \\
\qquad    \bnflit{LambdaExpression} \bnflit{.} ( ( \bnflit{parameter} \bnfnt{integer} ) \bnfor{} \bnflit{body} ) \\
\\
\bnfnt{member-reference} ::= \\
\qquad    \bnflit{MemberReference} \bnflit{.} ( \bnflit{qualifierExpression} \bnfor{} ( \bnflit{typeArgument} \bnfnt{integer} ) ) \\
\\
\bnfnt{member-select} ::= \\
\qquad    \bnflit{MemberSelect} \bnflit{.} \bnflit{expression} \\
\\
\bnfnt{method-invocation} ::= \\
\qquad    \bnflit{MethodInvocation} \bnflit{.} ( ( \bnflit{typeArgument} \bnfnt{integer} ) \bnfor{} \bnflit{methodSelect} \\
\qquad    \bnfor{} ( \bnflit{argument} \bnfnt{integer} ) ) \\
\\
\bnfnt{new-array} ::= \\
\qquad    \bnflit{NewArray} \bnflit{.} ( \bnflit{type} \bnfor{} ( \bnflit{dimension} \bnfor{} \bnflit{initializer} ) \bnfnt{integer} ) \\
\\
\bnfnt{new-class} ::= \\
\qquad    \bnflit{NewClass} \bnflit{.} ( \bnflit{enclosingExpression} \bnfor{} ( \bnflit{typeArgument} \bnfnt{integer} ) \bnfor{} \bnflit{identifier} \\
\qquad    \bnfor{} ( \bnflit{argument} \bnfnt{integer} ) \bnfor{} \bnflit{classBody} ) \\
\\
\bnfnt{parameterized-type} ::= \\
\qquad    \bnflit{ParameterizedType} \bnflit{.} ( \bnflit{type} \bnfor{} ( \bnflit{typeArgument} \bnfnt{integer} ) ) \\
\\
\bnfnt{parenthesized} ::= \\
\qquad    \bnflit{Parenthesized} \bnflit{.} \bnflit{expression} \\
\\
\bnfnt{return} ::= \\
\qquad    \bnflit{Return} \bnflit{.} \bnflit{expression} \\
\\
\bnfnt{switch} ::= \\
\qquad    \bnflit{Switch} \bnflit{.} ( \bnflit{expression} \bnfor{} ( \bnflit{case} \bnfnt{integer} ) ) \\
\\
\bnfnt{synchronized} ::= \\
\qquad    \bnflit{Synchronized} \bnflit{.} ( \bnflit{expression} \bnfor{} \bnflit{block} ) \\
\\
\bnfnt{throw} ::= \\
\qquad    \bnflit{Throw} \bnflit{.} \bnflit{expression} \\
\\
\bnfnt{try} ::= \\
\qquad    \bnflit{Try} \bnflit{.} ( \bnflit{block} \bnfor{} ( \bnflit{catch} \bnfnt{integer} ) \bnfor{} \bnflit{finallyBlock} \bnfor{} ( \bnflit{resource} \bnfnt{integer} ) ) \\
\\
\bnfnt{type-cast} ::= \\
\qquad    \bnflit{TypeCast} \bnflit{.} ( \bnflit{type} \bnfor{} \bnflit{expression} ) \\
\\
\bnfnt{type-parameter} ::= \\
\qquad    \bnflit{TypeParameter} \bnflit{.} \bnflit{bound} \bnfnt{integer} \\
\\
\bnfnt{unary} ::= \\
\qquad    \bnflit{Unary} \bnflit{.} \bnflit{expression} \\
\\
\bnfnt{union-type} ::= \\
\qquad    \bnflit{UnionType} \bnflit{.} \bnflit{typeAlternative} \bnfnt{integer} \\
\\
\bnfnt{variable} ::= \\
\qquad    \bnflit{Variable} \bnflit{.} ( \bnflit{type} \bnfor{} \bnflit{initializer} ) \\
\\
\bnfnt{while-loop} ::= \\
\qquad    \bnflit{WhileLoop} \bnflit{.} ( \bnflit{condition} \bnfor{} \bnflit{statement} ) \\
\\
\bnfnt{wildcard} ::= \\
\qquad    \bnflit{Wildcard} \bnflit{.} \bnflit{bound} \\
\\
\end{tabbing}


\subsection{Annotations\label{annotations-grammar}}

This section describes the details of how annotations are defined, how
annotations are used, and the different kinds of annotations in an annotation
file.


\subsubsection{Annotation definitions\label{annotation-definitions}}

An annotation definition describes the annotation's fields and their
types, so that they may be referenced in a compact way throughout the
annotation file. Any annotation that is used in an annotation file
% either on a program element or as a field of another annotation definition.
must be defined before use.
(This requirement makes it impossible to define, in an
annotation file, an annotation that is meta-annotated with itself.)
The two exceptions to this rule are the \code{@java.lang.annotation.Target} and
\code{@java.lang.annotation.Retention} meta-annotations. These meta-annotations
are often used in annotation definitions so for ease of use are they not required to
be defined themselves.
In the annotation file, the annotation definition appears within the
package that defines the annotation.  The annotation may be applied to
elements of any package.

Note that these annotation definitions should not be confused with the
\code{@interface} syntax used in a Java source file to declare an annotation. An
annotation definition in an annotation file is only used internally. An
annotation definition in an annotation file will often mirror an
\code{@interface} annotation declaration in a Java source file in order to use
that annotation in an annotation file.

% TODO, see https://github.com/typetools/annotation-tools/issues/25
% The Annotation File Utilities can read annotation definitions from the
% classpath, so it is optional to define them in the annotation file.

\begin{tabbing}
\qquad \= \kill
\bnfnt{annotation-definition} ::= \\
\qquad    \bnfcmt The \bnfnt{decl-annotation}s are the meta-annotations on this annotation. \\
\qquad    \bnflit{annotation} \bnflit{@}\bnfnt{name}
    \bnflit{:} \bnfnt{decl-annotation}* \lineend  \\
\qquad    \bnfnt{annotation-field-definition}* \\
\\
\bnfnt{annotation-field-definition} ::= \\
\qquad    \bnfnt{annotation-field-type} \bnfnt{name} \lineend \\
\\
\bnfnt{annotation-field-type} ::= \\
\qquad    \bnfcmt \bnfnt{primitive-type} is any Java primitive type (\code{int}, \code{boolean}, etc.). \\
\qquad    \bnfcmt These are described in detail in Section~\ref{types-and-values}. \\
\qquad    (\bnfnt{primitive-type} \bnfor{} \bnflit{String} \bnfor{} \bnflit{Class}
    \bnfor{} (\bnflit{enum} \bnfnt{name}) \bnfor{} (\bnflit{annotation-field} \bnfnt{name})) \bnflit{[]}? \\
\qquad        \bnfor{} \bnflit{unknown[]} \lineend
\end{tabbing}


\subsubsection{Annotation uses\label{annotation-uses}}

Java SE 8 has two kinds of annotations: ``declaration annotations'' and ``type
annotations''. Declaration annotations can be written only on method formal
parameters and the declarations of packages, classes, methods, fields, and local
variables. Type annotations can be written on any use of a type, and on type
parameter declarations. Type annotations must be meta-annotated
with \code{ElementType.TYPE\_USE} and/or \code{ElementType.TYPE\_PARAMETER}.
These meta-annotations are described in more detail in the JSR 308
specification~\cite{JSR308-webpage-201110}.

The previous rules have used two productions for annotation uses in an
annotation file: \bnfnt{decl-annotation} and \bnfnt{type-annotation}.
The \bnfnt{decl-annotation} and \bnfnt{type-annotation} productions use the same
syntax to specify an annotation. These two different rules exist only to show
which type of annotation is valid in a given location. A declaration annotation
must be used where the \bnfnt{decl-annotation} production is used and a type
annotation must be used where the \bnfnt{type-annotation} production is used.

The syntax for an annotation is the same as in a Java source file.

\begin{tabbing}
\qquad \= \kill
\bnfnt{decl-annotation} ::= \\
\qquad    \bnfcmt \bnfnt{annotation} must be a declaration annotation. \\
\qquad    \bnfnt{annotation} \\
\\
\bnfnt{type-annotation} ::= \\
\qquad    \bnfcmt \bnfnt{annotation} must be a type annotation. \\
\qquad    \bnfnt{annotation} \\
\\
\bnfnt{annotation} ::= \\
\qquad    \bnfcmt The name may be the annotation's simple name, unless the file \\
\qquad    \bnfcmt contains definitions for two annotations with the same simple name. \\
\qquad    \bnfcmt In this case, the fully-qualified name of the annotation name is required. \\
% TODO:
% Perhaps we could add that if a class is in the same package
% as an annotation it may always use the simple name (even if there's another
% annotation with the same simple name in another package)? - MP 06/28
\qquad    \bnflit{@}\bnfnt{name} [ \bnflit{(} \bnfnt{annotation-field} [ \bnflit{,} \bnfnt{annotation-field} ]+ \bnflit{)} ] \\
\\
\bnfnt{annotation-field} ::= \\
\qquad    \bnfcmt In Java, if a single-field annotation has a field named \\
\qquad    \bnfcmt ``\code{value}'', then that field name may be elided in uses of the\\
\qquad    \bnfcmt annotation:   ``\code{@A(12)}'' rather than ``\code{@A(value=12)}''. \\
\qquad    \bnfcmt The same convention holds in an annotation file. \\
\qquad    \bnfnt{name} \bnflit{=} \bnfnt{value}
\end{tabbing}

\noindent
Certain Java elements allow both declaration and type annotations (for example,
formal method parameters). For these elements, the \bnfnt{type-annotations}
rule is used to differentiate between the declaration annotations and the type
annotations.

\begin{tabbing}
\qquad \= \kill
\bnfnt{type-annotations} ::= \\
\qquad    \bnfcmt holds the type annotations, as opposed to the declaration annotations. \\
\qquad        \bnflit{type:} \bnfnt{type-annotation}* \lineend \\
\qquad        \bnfnt{compound-type}*
\end{tabbing}


\paragraph{Compound type annotations}

A compound type is a parameterized, wildcard, array, or nested type. Annotations
may be on any type in a compound type. In order to specify the location of an
annotation within a compound type we use a ``type path''. A
type path is composed one or more pairs of type kind and type argument index.

\begin{tabbing}
\qquad \= \kill
\bnfnt{type-kind} ::= \\
\qquad    \bnflit{0} \bnfcmt annotation is deeper in this array type \\
\qquad    \bnfor{} \bnflit{1} \bnfcmt annotation is deeper in this nested type \\
\qquad    \bnfor{} \bnflit{2} \bnfcmt annotation is on the bound of this wildcard type argument \\
\qquad    \bnfor{} \bnflit{3} \bnfcmt annotation is on the i'th type argument of this parameterized type \\
\\
\bnfnt{type-path} ::= \\
\qquad    \bnfcmt The \bnfnt{integer} is the type argument index. \\
\qquad    \bnfnt{type-kind} \bnflit{,} \bnfnt{integer} [ \bnflit{,} \bnfnt{type-kind} \bnflit{,} \bnfnt{integer} ]* \\
\\
\bnfnt{compound-type} ::= \\
\qquad    \bnflit{inner-type} \bnfnt{type-path} \bnflit{:} \bnfnt{annotation}* \lineend
\end{tabbing}

\noindent
The type argument index used in the \bnfnt{type-path} rule must be \bnflit{0} unless the \bnfnt{type-kind} is
\bnflit{3}. In this case, the type argument index selects which type argument
of a parameterized type to use.

\urldef\cftp\url|http://types.cs.washington.edu/jsr308/specification/java-annotation-design.html#class-file:ext:type_path|
Type paths are explained in more detail, with many examples to ease
understanding, in Section 3.4 of the JSR 308 Specification.\footnotemark
\footnotetext{\cftp}


\section{Example\label{example}}

Consider the code of Figure~\ref{fig:java-example}.
Figure~\ref{fig:annotation-file-examples} shows two legal annotation files
each of which represents its annotations.


\begin{figure}
\begin{verbatim}
package p1;

import p2.*; // for the annotations @A through @D
import java.util.*;

public @A(12) class Foo {

    public int bar;             // no annotation
    private @B List<@C String> baz;

    public Foo(@D("spam") Foo this, @B List<@C String> a) {
        @B List<@C String> l = new LinkedList<@C String>();
        l = (@B List<@C String>)l;
    }
}
\end{verbatim}
\caption{Example Java code with annotations.}
\label{fig:java-example}
\end{figure}


\begin{figure}
\begin{tabular}{|c|c|}
\hline
\begin{minipage}[t]{.5\textwidth}
\begin{verbatim}
package p2:
annotation @A:
    int value
annotation @B:
annotation @C:
annotation @D:
    String value

package p1:
class Foo: @A(value=12)

    field bar:

    field baz: @B
        inner-type 0: @C

    method <init>(
      Ljava/util/List;)V:
        parameter 0: @B
            inner-type 0: @C
        receiver: @D(value="spam")
        local 1 #3+5: @B
            inner-type 0: @C
        typecast #7: @B
            inner-type 0: @C
        new #0:
            inner-type 0: @C
\end{verbatim}
\end{minipage}
&
\begin{minipage}[t]{.45\textwidth}
\begin{verbatim}
package p2:
annotation @A
    int value

package p2:
annotation @B

package p2:
annotation @C

package p2:
annotation @D
    String value

package p1:
class Foo: @A(value=12)

package p1:
class Foo:
    field baz: @B

package p1:
class Foo:
    field baz:
        inner-type 0: @C

// ... definitions for p1.Foo.<init>
// omitted for brevity
\end{verbatim}
\end{minipage}
\\
\hline
\end{tabular}

\caption{Two distinct annotation files each corresponding to the code of
  Figure~\ref{fig:java-example}.}
\label{fig:annotation-file-examples}
\end{figure}


\section{Types and values\label{types-and-values}}

The Java language permits several types for annotation fields: primitives,
\code{String}s, \code{java.lang.Class} tokens (possibly parameterized),
enumeration constants, annotations, and one-dimensional arrays of these.

These \textbf{types} are represented in an annotation file as follows:
\begin{itemize}
\item Primitive: the name of the primitive type, such as \code{boolean}.
\item String: \code{String}.
\item Class token: \code{Class}; the parameterization, if any, is not
represented in annotation files.
\item Enumeration constant: \code{enum} followed by the binary name of
the enumeration class, such as \code{enum java.lang.Thread\$State}.
\item Annotation: \code{@} followed by the binary name of the annotation type.
\item Array: The representation of the element type followed by \code{[]}, such
as \code{String[]}, with one exception: an annotation definition may specify
a field type as \code{unknown[]} if, in all occurrences of that annotation in
the annotation file, the field value is a zero-length array.\footnotemark
\footnotetext{There is a design flaw in the format of array field values in a
class file.  An array does not itself specify an element type; instead, each
element specifies its type.  If the annotation type \code{X} has an array field
\code{arr} but \code{arr} is zero-length in every \code{@X} annotation in the
class file, there is no way to determine the element type of \code{arr} from the
class file.  This exception makes it possible to define \code{X} when the class
file is converted to an annotation file.}
\end{itemize}

Annotation field \textbf{values} are represented in an annotation file as follows:
\begin{itemize}
\item Numeric primitive value: literals as they would appear in Java source
code.
\item Boolean: \code{true} or \code{false}.
\item Character: A single character or escape sequence in single quotes, such
as \code{'A'} or \code{'\char`\\''}.
\item String: A string literal as it would appear in source code, such as
\code{"\char`\\"Yields falsehood when quined\char`\\" yields falsehood when quined."}.
\item Class token: The binary name of the class (using \code{\$} for
inner classes) or the name of the primitive type or \code{void}, possibly
followed by \code{[]}s representing array layers, followed by \code{.class}.
Examples: \code{java.lang.Integer[].class}, \code{java.util.Map\$Entry.class},
and \code{int.class}.
\item Enumeration constant: the name of the enumeration constant, such as
\code{RUNNABLE}.
\item Array: a sequence of elements inside \code{\char`\{\char`\}} with a comma
between each pair of adjacent elements; a comma following the last element is
optional as in Java.  Also as in Java, the braces may be omitted if the
array has only one element.
Examples: \code{\char`\{1\char`\}}, \code{1},
\code{\char`\{true, false,\char`\}} and \code{\char`\{\char`\}}.
\end{itemize}

The following example annotation file shows how types and values are represented.

\begin{verbatim}
package p1:

annotation @ClassInfo:
    String remark
    Class favoriteClass
    Class favoriteCollection // it's probably Class<? extends Collection>
                             // in source, but no parameterization here
    char favoriteLetter
    boolean isBuggy
    enum p1.DebugCategory[] defaultDebugCategories
    @p1.CommitInfo lastCommit

annotation @CommitInfo:
    byte[] hashCode
    int unixTime
    String author
    String message

class Foo: @p1.ClassInfo(
    remark="Anything named \"Foo\" is bound to be good!",
    favoriteClass=java.lang.reflect.Proxy.class,
    favoriteCollection=java.util.LinkedHashSet.class,
    favoriteLetter='F',
    isBuggy=true,
    defaultDebugCategories={DEBUG_TRAVERSAL, DEBUG_STORES, DEBUG_IO},
    lastCommit=@p1.CommitInfo(
        hashCode={31, 41, 59, 26, 53, 58, 97, 92, 32, 38, 46, 26, 43, 38, 32, 79},
        unixTime=1152109350,
        author="Joe Programmer",
        message="First implementation of Foo"
    )
)
\end{verbatim}


\section{Alternative formats\label{alternative-formats}}

We mention multiple alternatives to the format described in this document.
Each of them has its own merits.
In the future, the other formats could be implemented, along with tools for
converting among them.
% Then, we can see which of the formats programmers prefer in practice.



An alternative to the format described in this document would be XML\@.
% It would be easy to use an XML format to augment the one proposed here, but
XML does not seem to provide any compelling advantages.  Programmers
interact with annotation files in two ways:  textually (when reading, writing,
and editing annotation files) and programmatically (when writing
annotation-processing tools).  Textually, XML can be
very hard to read; style sheets mitigate this
problem, but editing XML files remains tedious and error-prone.
Programmatically, a layer of abstraction (an API) is needed in any event, so it
makes little difference what the underlying textual representation is.
XML files are easier to parse, but the parsing code only needs to be
written once and is abstracted away by an API to the data structure.


Another alternative is a format like the \code{.spec}/\code{.jml} files
of JML~\cite{LeavensBR2006:JML}.  The format is similar to Java code, but
all method bodies are empty, and users can annotate the public members of a
class.  This is easy for Java programmers to read and understand.  (It is a
bit more complex to implement, but that is not particularly germane.)
Because it does not permit complete specification of a class's annotations
(it does not permit annotation of method bodies), it is not appropriate for
certain tools, such as type inference tools.  However, it might be desirable
to adopt such a format for public members, and to use the format
described in this document primarily for method bodies.


The Checker Framework~\cite{DietlDEMS2011,CF} uses two additional formats for
annotations. The first format is called ``stub files.'' A stub file is similar
to the \code{.spec}/\code{.jml} files described in the previous paragraph. It
uses Java syntax, only allows annotations on method headers and does not require
method bodies. A stub file is used to add annotations to method headers of
existing Java classes. For example, the Checker Framework uses stub files to add
annotations to method headers of libraries (such as the JDK) without modifying
the source code or bytecode of the library. A single stub file can contain
multiple packages and classes. This format only allows annotations on method
headers, not class headers, fields, and method bodies like in a \code{.jaif}
file. Further, stub files are only used by the Checker Framework at run time,
they cannot be used to insert annotations into a source or classfile.


The Checker Framework also uses a format called an ``annotated JDK.'' The
annotated JDK is a \code{.jar} file containing the JDK with annotations. It is
created with the Annotation File Utilities, but the annotations are stored in a
format similar to a stub file, instead of in a \code{.jaif} file. The annotated
JDK starts with a source file for each file in the JDK to be annotated. Like a
stub file, each source file only contains method headers with annotations. The
annotated JDK also supports annotations in the class header. To build the
annotated JDK \code{.jar} file, the source files are compiled, then the
\code{extract-annotations} script is run on them to generate a \code{.jaif} file
for each source file. The \code{insert-annotations} script then inserts the
annotations contained in each \code{.jaif} file into the corresponding JDK class
file. These are then packaged up into a single \code{.jar} file. Like a stub
files, the annotated JDK is easier to read and write since it uses Java syntax.
However, the annotated JDK requires a different file for each original Java
source file. It does not allow annotations on fields and in method bodies. The
annotated JDK also only contains annotations in the JDK and not other Java
files.



\bibliographystyle{alpha}
\bibliography{annotation-file-format,bibstring-unabbrev,types,ernst,invariants,generals,alias}

\end{document}

% LocalWords:  java javac OuterClass InnerClass TODO Kleene MP subannotations
% LocalWords:  enum arr quined int pt instanceof RUNTIME JVML ILjava boolean
% LocalWords:  programmatically jml ernst jaif whitespace 0pt decl enums
%  LocalWords:  filenames typeparam javap init clinit ast un lowercased io
%  LocalWords:  ExpressionStatement AnnotatedType underlyingType ArrayType
%  LocalWords:  ArrayAccess leftOperand rightOperand CompoundAssignment
%  LocalWords:  ConditionalExpression trueExpression falseExpression
%  LocalWords:  DoWhileLoop EnhancedForLoop ForLoop thenStatement NewArray
%  LocalWords:  elseStatement InstanceOf LabeledStatement LambdaExpression
%  LocalWords:  MemberReference qualifierExpression typeArgument NewClass
%  LocalWords:  MemberSelect MethodInvocation methodSelect classBody
%  LocalWords:  enclosingExpression ParameterizedType finallyBlock AScene
%  LocalWords:  TypeCast UnionType typeAlternative WhileLoop ElementType
%  LocalWords:  AClass AMethod AElement objectweb anno tations parseScene
%  LocalWords:  CriterionList isSatisifiedBy CriteriaList afu getPositions
%  LocalWords:  InPackageCriterion InClassCriterion InMethodCriterion
%  LocalWords:  ParamCriterion inserter RUNNABLE ASM src asm
