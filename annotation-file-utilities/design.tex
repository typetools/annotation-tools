\documentclass{article}
\title{AFU v4 Design}
\author{Dan Brown}

\usepackage{graphics}
\usepackage{verbatim}
\newenvironment{code}{\footnotesize\verbatim}{\endverbatim\normalsize}
\newenvironment{jcode}{\footnotesize\verbatim}{\endverbatim\normalsize}

\begin{document}
\maketitle

\section{Design Considerations}

The Annotation File Utilities (AFU) include programs for extracting
annotations from Java class files and inserting annotations into Java
source or class files, in accordance with the Java 8 language and virtual
machine specification.  This
document uses AFU in the singular, in reference to the source
insertion program except where otherwise noted.

The AFU suffers from multiple design flaws that impede performance and
maintainability, including:

\begin{description}
\item[Unstructured text modification.]  The AFU inserts
annotations into source text instead of modifying syntax trees.
The use of source positions rather than tree paths for indicating
insertion placement has led to code that is fragile and hence difficult
to maintain.
\item[Non-uniform logic.]  There are two distinct ways to
identify where to insert annotations into the source, each having its
own distinct data structures and logic.  Consequently, it is possible to
have inconsistent results for different representations of the same
location.
\item[Inefficient traversal.]  The search strategy is
generate-and-test: it chooses an AST node in a preorder traversal and
checks every insertion specification relevant to the class (or package)
to determine whether it applies to the current node.  Furthermore, since
the implementations of some tests contain recursive calls going up the
path to the root, these tests can be repeated many times over for some
nodes.
\end{description}

The proposed design is aimed at mitigating these problems.  It makes
the traversal a single pass over the annotations, with limited
re-visitation of AST nodes.  It reduces the chance for inconsistency
between formats by finding the AST node that corresponds to a code
location given in the scene before determining the insertion position.
It modifies an AST data structure rather than the source itself,
preserving formatting and commentary by incorporating non-leaf text
(i.e.\ whitespace and comments) into the representation of the AST.

\section{Current AFU}

The current AFU source code (3.6.16 as of 11 June 2015) is organized
into four packages (not counting the \texttt{asmx} and
\texttt{scene\_lib} sub-projects):
\begin{itemize}
\item  \texttt{annotator}, containing the main class and a class
representing source text;
\item  \texttt{annotator.find},\ containing the classes that implement
the core insertion logic;
\item  \texttt{annotator.scanner},\ containing support classes for
\texttt{annotator.find};\ and
\item  \texttt{annotator.specification},\ containing classes that
manage intermediate representations of the insertions specified in a
Java Annotation Index File (JAIF).
\end{itemize}
Should the proposed redesign be implemented, it is anticipated that the
first will be simplified somewhat and modified to reflect changes in the
other packages, the second will be radically simplified, the third will
remain the same, and the fourth will be eliminated.

The details of \texttt{annotator.scanner}\ are not discussed here, since
the proposed redesign will neither change them nor be influenced by
them.  Of the classes in \texttt{annotator.specification},\ only
\texttt{IndexFileSpecification},\ which extracts \texttt{Insertion}s
from \texttt{scene-lib}'s JAIF representation
(\texttt{scenelib.annotations.el.AScene}), is relevant to the redesign.

By far the largest and most complex package is \texttt{annotator.find}.
The insertion logic is distributed among \texttt{TreeFinder} and
numerous implementations of the \texttt{Criterion} and
\texttt{Insertion} interfaces.  (A \texttt{Criterion} is a reified
condition that provides a method
\texttt{boolean isSatisfiedBy(TreePath path)},
and an \texttt{Insertion} includes matching criteria, output parameters,
and methods for generating the source text to be inserted.)

A simplified outline of the basic flow of the program is as follows:

\begin{enumerate}
\item  Interpret each JAIF as a ``scene'' (\texttt{AScene}), then invoke
the \texttt{parse} method in \texttt{IndexFileSpecification} to extract
the indicated \texttt{Insertion}s from the scene.
\item  Parse each Java source file to obtain a (type-checked) AST.
\item  Create an empty map from (Integer) source positions to
insertions.
\item  For each node in the AST in pre-order:
\begin{itemize}
\item  For each extracted \texttt{Insertion} that is relevant to the
immediately enclosing class (or package for compilation unit nodes):
\begin{itemize}
\item  If the path from the root to the current node satisfies every
\texttt{Criterion} for the \texttt{Insertion}:
\begin{enumerate}
\item  Find the appropriate source position for the annotation.
\item  Add a mapping from the position to the \texttt{Insertion} to the
map.
\item  Remove the insertion from the extracted set.
\end{enumerate}
\end{itemize}
\end{itemize}
\item  For each position-to-insertion mapping in descending order by
position:
\begin{itemize}
\item  Generate insertion text and insert at indicated position.
\end{itemize}
\item  Emit modified source code.
\end{enumerate}

\section{Proposed Revisions}

\begin{figure}[ht]
\begin{center}
\resizebox{\columnwidth}{!}{\includegraphics{dataflow.png}}
\end{center}
\caption{
Data flow between components.  Boxes represent sources and sinks;
ellipses represent Java objects.  Bold lines mark new classes and data
paths, and dashed lines indicate current data flows to be bypassed.
}
\label{fig:dataflow}
\end{figure}

Figure~\ref{fig:dataflow} gives an overview of proposed changes to the
flow of data between classes.  First, the current re-interpretation of
the scene as a ``specification'' and subsequent extraction of
\texttt{Criteria} will be eliminated, as the necessary information can
be determined directly from the \texttt{Scene}.  Second, the source text
will be incorporated into the augmented AST in a more convenient form,
so there will no need to refer back to the source itself.

The division of responsibilities among classes will remain largely the
same, with the major differences corresponding to the changes in the
data flow.  First, the annotation locations will be read from the
\texttt{Scene} and mapped to the corresponding path in the (augmented)
AST (if any).  Second, since \texttt{AAST} will incorporate the salient
capabilities of \texttt{JCTree} and \texttt{Source}, it will cover the
responsibilities of both classes.

Many of the current supporting classes will be retained as they are.  In
particular, the new AFU will continue to rely on javac's AST classes and
parser, and \texttt{scene-lib} probably will need little or no
modification.  The \texttt{annotator.find} package will be radically
simplified with the elimination of the \texttt{Criterion} and
(possibly) \texttt{Insertion} hierarchies, as the insertion location
within the AST can be determined directly from the \texttt{Scene}.

The redesign can be broken down into two independent parts:
\begin{description}
\item{Path-based positioning.} Eliminate the \texttt{Criterion}
hierarchy; determine the insertion location and the text to be inserted
during (a single) scene traversal; let the scene traversal rather than
an AST traversal be the outer loop.  Advantages: since insertion is
based on the node rather than a numeric position, there is only one
positioning logic (even if there are still two matching objects); an
entire pass and a large amount of intermediate data are eliminated, thus
improving time and space efficiency.
\item{Tree transformation.}  Eliminate the \texttt{Insertion} hierarchy;
instead of calculating numeric positions, apply a transformation to the
appropriate subtree.  Advantage: the logic of tree transformation is
both closer to the specification and easier to comprehend (and maintain)
than the logic of finding numeric positions for inserting text by
reverse order.
\end{description}

While these have no obvious technical disadvantages with respect to the
existing AFU, the payoff must be weighed against the costs.  The current
codebase is mostly stable at this point, so any increase in reliability
will be small (though for verification software, a jump from, say,
99.99\% to 99.9999\% correct makes an enormous difference).
Fortunately, the AFU's extensive test suite will make it clear how
quickly a revised implementation achieves the same level of reliability.

\subsection{Path-based Positioning}

The current AFU creates transforms the scene into a set of
\texttt{Insertions}, each with its own \texttt{Criteria} (a collection
of objects of various \texttt{Criterion} subclasses).  In doing so, it
throws away information about hierarchical relationships that could be
used to guide the search for matching locations in the code.

\begin{figure}[ht]
\begin{center}
\resizebox{156pt}{!}{\includegraphics{corresp.png}}
\end{center}
\caption{
Correspondence between scene and AST.
}
\label{fig:corresp}
\end{figure}

It makes more sense to traverse an AST in parallel (conceptually, not
necessarily by different processors) with a scene that refers to it.
For example, if a class specification (\texttt{AClass}) in a scene
corresponds to a class node in an AST, each time one of the method
specifications (\texttt{AMethod}) is visited, the corresponding method
node (see Figure~\ref{fig:corresp}) should be visited as well to see
whether any insertions under the \texttt{AMethod} apply to it.  This
strategy not only obviates an additional pass but avoids much useless
test repetition, as there is no need to test \emph{every} potential
insertion against \emph{every} AST node encountered before the correct
insertion site.  With this change, it becomes possible to generate
\texttt{Insertion}s on the fly---or perhaps even to go ahead and insert
text, though managing order dependencies would take some additional
work.

The main loop of the program thus will become a pre-order traversal of
\texttt{AScene} rather than of \texttt{JCTree}, although the scene
traversal will explore corresponding parts of the AST.  In other words:
the process will be guided by the annotations to be inserted rather than
by the shape of the AST.  Hence, each specified annotation will be
visited exactly once.  The information currently used to create
\texttt{Criteria} for the insertions can instead be used to directly
determine the path (if any) in the AST to which the current scene
element corresponds.

\subsection{Tree Transformation}

The current AFU finds numeric source positions in trees and inserts
annotations directly into the source code.  The position-finding logic
is therefore applied at a time other than insertion, introducing
insertion order dependencies and making it harder to find the source of
a bug.

A new class hierarchy, the ``Augmented AST'' (\texttt{AAST}), will
directly manage non-leaf text--that is, inter-leaf text (whitespace and
comments) and intra-branch text (implicit or contained in branch node
data members).  It will be an implementation of
\texttt{com.sun.source.Tree} that provides needed bits from
\texttt{com.sun.tools.javac.tree.JCTree} and defines a parallel subclass
hierarchy.  AAST and subclasses provide methods to retrieve the
inter-leaf text segments that precede and follow an AST node and
printing methods that preserve the formatting of the source code.

Non-leaf text will be stored in three types of locations in an AAST:
\begin{enumerate}
\item  Each leaf node will store any non-code text that immediately
follows.
\item  For branch nodes, for every implicit text segment or data member
that is represented in the source code but not in a descendant node, the
node will store any whitespace and comments that follow, unless the
content is at the end of the node's text.
\item  The computation unit node will store any text that precedes the
beginning of the Java code.
\end{enumerate}
In this way, all text segment references occur exactly once in the AST,
and annotations can be inserted at the front with no need to take
preceding text into account.

An annotation insertion will consist of a transformation of a tree node.
If no additional code is to be generated, the transformation is simple:
\begin{itemize}
\item  For type annotations, the parent AST node of the element to be
annotated adds the optional annotation node if it is absent, then adds
the annotation to the annotation node.
\item  For declaration annotations, the annotation is added onto the
front of a declaration's list of modifiers.
\end{itemize}
Cases that involve code generation are more complex.
Figures~\ref{fig:nocast} and \ref{fig:typecast} give an example (pre-
and post-insertion, respectively) of a transformation that requires a
typecast to be generated and inserted.  Note that previously existing
non-leaf text segments remain unaltered.

\begin{figure}[ht]
\begin{center}
\resizebox{261pt}{!}{\includegraphics{nocast.png}}
\end{center}
\caption{
Augmented subgraph for \texttt{int i = 0}.  Text of leaf nodes and text
representing elements of branch nodes are connected with solid arrows,
and inter-leaf and inter-element text segments are connected with dashed
arrows.  The text of the subgraph can be read from the lower frontier of
the graph.  (Underscores represent spaces here.)
}
\label{fig:nocast}
\end{figure}

\begin{figure}[ht]
\begin{center}
\resizebox{\columnwidth}{!}{\includegraphics{typecast.png}}
\end{center}
\caption{
Augmented subgraph for \texttt{int i = (@A int) 0}, the result of adding
an annotated cast \texttt{(@A int)} to the previous figure's code.  With
respect to the graph in the previous figure, \texttt{TypeCast} has taken
\texttt{Literal}'s place in the tree, and \texttt{Literal} has become a
child of \texttt{TypeCast} along with the newly inserted material.
}
\label{fig:typecast}
\end{figure}

The other major change will be in the output generation process, which
will depend on the \texttt{AAST} rather than directly on the source.
The \texttt{AAST}'s pretty-printer will perform an in-order traversal
and emit the text of each leaf interleaved with the non-leaf text.

\subsection{Algorithm Summary}

\begin{enumerate}
\item  Interpret each JAIF as a ``scene'' (\texttt{AScene}).
\item  Parse each Java source file to obtain a (type-checked) AST.
\item  For each AST:
\begin{enumerate}
\item  Transform the AST into an AAST.
\item  For each annotated element in the scene in pre-order:
\begin{itemize}
\item  Try to find the AAST node to which the element corresponds.
\item  If the node exists (or can be generated):
\begin{itemize}
\item  Transform the subtree rooted at the node to incorporate the
annotation.
\end{itemize}
\end{itemize}
\item  Pretty-print AAST.
\end{enumerate}
\end{enumerate}

\section{Planning}

There will be some time required for integrating the two parts if they
are implemented separately.  In particular, doing tree transformation
separately from path-based positioning means reifying transformations
so they can be done in a second pass.  Hence it makes more sense to
do path-based positioning on the way to doing tree transformation.

Estimated times are as follows:
\begin{itemize}
\item  Path-based positioning: 2.5 weeks.
\item  Tree transformation: 3 to 3.5 weeks if done after path-based
positioning, 4 weeks otherwise.
\end{itemize}
Thus, doing the full job should take about six weeks.

\appendix

\section{API extensions (non-exhaustive)}

\begin{jcode}
public abstract class AAST extends com.sun.source.Tree {
  /** Interleaves non-leaf text with text serialization of AAST. */
  public void prettyPrint(java.io.Writer w) throws java.io.IOException;

  /** Insert single annotation at given path. */
  public insertAnnotation(TreePath path, Annotation anno);
}

// main loop
AAST insertAnnotations(scenelib.annotations.el.AScene scene, AAST aast);
\end{jcode}

\newpage
\section{Excerpts from original API}\label{sec:exc}

\begin{jcode}
public interface Tree {
    public enum Kind {  // MUCH larger in actual interface
        /**
         * Used for instances of {@link BinaryTree} representing
         * addition or string concatenation {@code +}.
         */
        PLUS(BinaryTree.class),
        /**
         * Used for instances of {@link LiteralTree} representing
         * an integral literal expression of type {@code int}.
         */
        INT_LITERAL(LiteralTree.class);
    }
    Kind getKind();
    /**
     * Accept method used to implement the visitor pattern.  The
     * visitor pattern is used to implement operations on trees.
     *
     * @param <R> result type of this operation.
     * @param <D> type of additional data.
     */
    <R,D> R accept(TreeVisitor<R,D> visitor, D data);
}
public interface ExpressionTree extends Tree {}
public interface BinaryTree extends ExpressionTree {
    ExpressionTree getLeftOperand();
    ExpressionTree getRightOperand();
}
public interface LiteralTree extends ExpressionTree {
    Object getValue();
}

public abstract class JCTree implements Tree {
    public int getStartPosition() {
        return TreeInfo.getStartPos(this);
    }
    public static abstract class JCExpression
            extends JCTree implements ExpressionTree {}
    public static class JCBinary extends JCExpression implements BinaryTree {
        private Tag opcode;
        public JCExpression lhs;
        public JCExpression rhs;
        public Symbol operator;
        protected JCBinary(Tag opcode,
                         JCExpression lhs,
                         JCExpression rhs,
                         Symbol operator) {
            this.opcode = opcode;
            this.lhs = lhs;
            this.rhs = rhs;
            this.operator = operator;
        }
        public Kind getKind() { return TreeInfo.tagToKind(getTag()); }
        public JCExpression getLeftOperand() { return lhs; }
        public JCExpression getRightOperand() { return rhs; }
        public Symbol getOperator() { return operator; }
        @Override
        public <R,D> R accept(TreeVisitor<R,D> v, D d) {
            return v.visitBinary(this, d);
        }
    }
    public static class JCLiteral extends JCExpression implements LiteralTree {
        public TypeTag typetag;
        public Object value;
        protected JCLiteral(TypeTag typetag, Object value) {
            this.typetag = typetag;
            this.value = value;
        }
        public Kind getKind() {
            return typetag.getKindLiteral();
        }
        public Object getValue() {
            switch (typetag) {
                case BOOLEAN:
                    int bi = (Integer) value;
                    return (bi != 0);
                case CHAR:
                    int ci = (Integer) value;
                    char c = (char) ci;
                    if (c != ci)
                        throw new AssertionError("bad value for char literal");
                    return c;
                default:
                    return value;
            }
        }
        @Override
        public <R,D> R accept(TreeVisitor<R,D> v, D d) {
            return v.visitLiteral(this, d);
        }
    }
}
\end{jcode}

\end{document}


%%  LocalWords:  AFU asmx JAIF IndexFileSpecification TreeFinder boolean
%%  LocalWords:  isSatisfiedBy TreePath AScene AAST JCTree javac's subtree
%%  LocalWords:  AFU's AClass AMethod subgraph TypeCast prettyPrint anno
%%  LocalWords:  insertAnnotation insertAnnotations aast BinaryTree opcode
%%  LocalWords:  LiteralTree getKind TreeVisitor ExpressionTree getValue
%%  LocalWords:  getLeftOperand getRightOperand getStartPosition TreeInfo
%%  LocalWords:  getStartPos JCExpression JCBinary lhs rhs tagToKind ci
%%  LocalWords:  getTag getOperator visitBinary JCLiteral TypeTag typetag
%%  LocalWords:  getKindLiteral AssertionError visitLiteral
